\documentclass[12pt]{report}
\usepackage[utf8]{inputenc}
\usepackage{amsmath, amsfonts}
\usepackage{geometry}
\usepackage{hyperref}
\usepackage{titling}
\usepackage{listings}
\usepackage{setspace}
\usepackage{graphicx}
\usepackage{caption}

\title{Technical Constraints}
\author{Muhammad Azeem Haider - Ali Asghar Yousuf \\
        Muhammad Shahid Mehmood - Musab Sattar} 
\date{\today}

\begin{document}
\maketitle

\newpage

\section*{Architecture Decisions}

Thekedaar hatao is a unique mobile application with the aim to remove the middleman (thaikaydaar) during the construction process.
The application will have quite a number of technical features that will be addressed in the subsequent sections. The application is being made using two professional frameworks. 

\begin{itemize}

        \item Flutter 
        \item NodeJS

\end{itemize}

Lets dive into our reasoning for choosing Flutter and NodeJS. 

\subsubsection*{Flutter}

Flutter is a high quality framework that makes it easy to build a mobile application swiftly. Since the application will be compatible with both Android and IOS,
Flutter seems like the obvious and the best choice for our Frontend development. Flutter can run in any browser, which is why it provides us with the chance to work with NodeJS without creating many hurdles.  

\subsubsection*{NodeJS}

NodeJS was an executive decision that the group members all agreed on. The plan was to first use firebase for Backend development but since we want to work with the Database and the server, 
NodeJS seems like a good fit for the application and the team. NodeJS when combined with Flutter for development offers some excellent results when developing an application. 

\section*{Compatibility}

We want to keep our mobile application simple and efficient. For this purpose, while we will not be adding any flashy feature, we will want our application to run smoothly. For this purpose we would advice the users to use a mobile phone with an operating system of Android 10 known as Android Quince Tart for Android phones and for apple users to have IOS 15 for optimum use. The application will be supported by both the Operating systems and tested on both the operating systems to make sure that users at the time of release face no unwanted problems. 

\section*{Application Features}

The mobile application idea for Thaikaydaar Hatao is simple yet elegant. The application will have three major features that are as followed;

\begin{itemize}

        \item Built in calculator 
        \item Forum to interact with other users
        \item Marketplace to Buy and Sell

\end{itemize}

\subsubsection*{Built in Calculator}

The idea for the built in calculator stems from the frustration that people go through when the middleman provides them with bad advice prompting them to buy more material than first needed. The calculator will be added which will ask the user to add the land in sq/ft, and how many floors one is planning on building. Once the user adds the sq/ft of land and the floor that they are planning on building, the calculator will provide them with an estimate raw material amount needed. This will include cement, cinder block, sand and other such requirements needed to build a house. 

% Can also add the cost of this material 

\subsubsection*{Forum}

Forum will be a place for advice where the users can ask and advice others on problems that have arisen. The forum will be there for users to connect with each other and will provide an option to talk in private with each other as well. 

\subsubsection*{MarketPlace}

The marketplace will be available on the application for users to sell the extra material left once their house is finished. This will also be beneficial for people looking for small amounts of material to buy for little construction around the house or if the construction is almost complete and only some amount of material is needed. 

\end{document}
