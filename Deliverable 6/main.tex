\documentclass[title page]{article}
%%%%%%% PACKAGES %%%%%%%%
\usepackage[utf8]{inputenc}
\usepackage{graphicx}
\usepackage{geometry}
\geometry{a4paper, margin=1in}
\usepackage{hyperref}
\hypersetup{
    colorlinks=true,
    linkcolor=blue,
    filecolor=magenta,      
    urlcolor=cyan,
}

\title{Software Engineering Project Test Strategy}
\author{Muhammad Azeem Haider $\mid$ Ali Asghar Yousuf \\
      Mohammad Shahid Mahmood $\mid$ Musab Sattar}
\date{\today}

\begin{document}
\pagenumbering{roman} % Start roman numbering
\clearpage\maketitle
\thispagestyle{empty}
    % \begin{center}
    %     \vspace*{\fill}
    %     \Large{Thekedaar Hatao \\
    %     Babar Azam Hamood CS}
    %     \vspace*{\fill}
    % \end{center}
\newpage
\setcounter{page}{1}
\tableofcontents


\newpage
\pagenumbering{arabic} % Start roman numbering

%%% CONTENT HERE %%%%

\maketitle

\section{Scope}
\subsection{Introduction}

Thekedaar Hatao is an app that gives an estimate of the amount of raw material that might be needed to create your house based on how much land the house is to be built on. It provides several functions such as login, sign up, forum to communicate on, material estimator, and marketplace.

The objective of our product is to allow users, the best necessary tools, conditions, and work-specific thought calculations to ensure they achieve what they plan to accomplish. The app is proposed to people wanting to establish a new area of residence for current or future planning. 

\subsection{Testing Area}
In the testing phase, we aim to test as many possible areas as we can, 



\section{Test Approach}

\subsection{Testing process}
The process of testing our app will majorly depend on manual and automated testing. We will also conduct a beta test within the Habib community

\subsection{Work Division}
\begin{enumerate}
    \item \textbf{Manual testing:} Musab Sattar and Azeem Haider
    \item\textbf{Automated testing:} Ali Asghar and Shahid Mehmood
\end{enumerate}

\subsection{Types of testing}

\subsubsection{Load testing}
We will be using the following to test our applications load endurance;
\begin{itemize}
    \item Jmeter
    \item Loadster
    \item Ladrunner
\end{itemize}

\subsubsection{Security testing}
We will perform some low-level security tests using SQL injections.
\subsubsection{Performance testing}
We will be performing manual performance testing amongst the four members of the group. 

\subsubsection{Unit testing}
The following are important points to keep in mind regarding the unit testing;
\begin{itemize}
    \item Unit tests will be written by the developers of the group project before merging the code into the main branch to use for unit testing.
    \item These developers will also be looking after the smooth running of the unit tests that they have submitted.
\end{itemize}

\subsubsection{Integration testing}
Integration tests will be carried out manually. This will be done by the team leading the manual testing for our mobile application. 

\subsection{Approach \& Automation Tools}
We will be using \textbf{PyTest} for automated testing as it is the most proficient for the use at the moment. 


\subsection{Bug Monitoring}
\textbf{Jira} will be our main tool for bug tracking and monitoring during and before the release of the application. 


\section{Test Platform \& Environment}
The following test platforms will be used to ensure that our application does not fail at the time of release. These platforms and environments we believe are ideal for properly testing our application. 

\subsection{Test Platform}
Since our backend will be developed using \texttt{Django} framework we will perform our automated testing using \texttt{PyTest}.

\subsection{Test Environment}
\begin{itemize}
    \item \textbf{Operating System:} Android 12 and IOS 15
    \item \textbf{Browser:} Chrome, Windows 10, Windows 11
\end{itemize}

\section{Release Control}
We will use Git as our version control system. Code changes will be reviewed and approved by a code reviewer before merging into the main branch. The team will give the "go" signal once the code reviewer has checked for any potential bugs. 

\section{Risk Analysis}
\begin{itemize}
    \item Data Security Risks
    \item Performance Issues 
    \item Compatability issues 
    \item Code violations threatening stability 
    \item Resource Constraints
\end{itemize}

\section{Test Plan}
The test plan is extremely important to give a shape to the scope of testing. It also explains the objectives of testing and what types of tests the team has to perform to get the correct results given the limited amount of resources. 

\subsection{Test Case ID}
Each test case will be assigned a unique identifier, such as a number or a code given by the testing team, that distinguishes it from other test cases. This allows for easy tracking of the test case's status and results. This will help in identifying any issues or failures that occur during testing.

\subsection{Test Case Name}
The test case name should be a clear and concise description of the functionality or feature being tested. This will accurately reflect the purpose of the test case and make it easy for testers to understand what is being tested.

\subsection{Test Case Objective}
The objective of the test case will be defined before writing the test case or running it, so the development team knows what metrics the test case has to be developed and what metrics the applications has to pass in order for the test case to be passed. 

\subsection{Prerequisites}
The prerequisites section outlines any conditions or requirements that must be met before the test case can be executed. This may include setting up test data, configuring the testing environment, or executing other tests first.

It is important to clearly define the prerequisites to ensure that the test case is executed correctly and that accurate results are obtained.

\subsection{Test Steps}
The test steps should outline the sequence of actions to be taken to execute the test case. This includes any inputs that need to be provided, any buttons that need to be clicked, or any pages that need to be navigated to.

\subsection{Expected Results}
The expected results should clearly state what outcome is expected from the test case. This may include displaying specific information on the screen, generating an error message, or navigating to a different page.

It is important to clearly define the expected results to ensure that the tester knows what to look for when executing the test case, and to ensure that the test results are meaningful and accurate.

\end{document}
